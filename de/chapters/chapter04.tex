\chapter{Herleitung der Compton-Formel}
\setcounter{section}{4}
\setcounter{subsection}{0}
\setcounter{subsubsection}{1}
\setcounter{secnumdepth}{3}
\setlength{\parindent}{0pt}
% Boxen-Stile definieren



%\section{Herleitung der Compton-Formel}
\subsection*{Ein Photon stößt auf ein Elektron}


\begin{center}
	\begin{tikzpicture}[scale=1.5, >=latex]
		
		% Achsen
		\draw[->] (-0.2,0) -- (3.5,0) node[right] {$x$};
		\draw[->] (0,-2.2) -- (0,2.2) node[above] {$y$};
		
		% Ursprung = Elektron vor Stoß
		\filldraw[blue] (0,0) circle (0.04) node[below left] {Elektron (ruhend)};
		
		% Einfallender Photon
		\draw[->, thick, blue] (-2,0) -- (0,0);
		\node[blue] at (-2.2, 0.2) {Photon einfallend $\vec{p}_\gamma$};
		
		% Elektron nach Stoß
		\draw[->, thick, blue] (0,0) -- (2,-1);
		\node[blue] at (0.8,-0.6) {$\vec{p'}_e$};
		\node[blue] at (2.2,-1.1) {Elektron nach Stoß};
		
		% Photon nach Stoß
		\draw[->, thick, red] (0,0) -- (1.3,2) ;
		\node[red] at (0.6,1.3){$\vec{p'}_\gamma$};
		\node[red] at (1.5,2.1) {Photon nach Stoß};
		
		% Winkel Theta (zwischen x-Achse und gestreutem Photon)
		%\draw (1,0) arc (0:-26.565:1);
		%\node at (1.2,-0.3) {$\theta$};
		
		% Optional: Winkel des Elektrons (phi)
		\draw (0.7,0) arc (0:57.995:0.7);
		\node at (0.75,0.3) {$\theta$};
		\node[blue] at (-0.2,0.2) {$\vec{p}_e$};
	\end{tikzpicture}
\end{center}



\subsection{Energieübertragung beim Compton-Effekt}

Das einfallende Photon besitzt eine höhere Energie (und damit eine höhere Frequenz) als das gestreute Photon nach dem Stoß. Ein Teil der Photonenenergie wird auf das Elektron übertragen, wodurch das Photon Energie verliert. Da die Energie eines Photons proportional zur Frequenz ist,
\[
E = h \nu,
\]
verringert sich nach dem Stoß die Frequenz des gestreuten Photons – seine Wellenlänge wird entsprechend größer.
\begin{DidacticBox}[Energie vs. Frequenz]
	Ein Photon verliert beim Stoß Energie \(\Delta E\); da \(E=h\nu\), sinkt seine Frequenz und die Wellenlänge wächst (\(\lambda' > \lambda\)). 
	Das ist die zentrale Intuition hinter der späteren Verschiebung \(\Delta\lambda\).
\end{DidacticBox}

\subsection{Die Energieerhaltung bei dem Stoß}
\[
E_\gamma + E_e = E'_\gamma + E'_e
\]

\begin{align*}
	E_\gamma :&\quad \text{Energie des einfallenden Photons} \\
	E_e       : &\quad \text{Energie des Elektrons vor dem Stoß } \\
	E'_\gamma  :&\quad \text{Energie des gestreuten Photons} \\
	E'_e       :&\quad \text{Energie des Elektrons nach dem Stoß }
\end{align*}
Zur Vereinfachung der Schreibweise wird definiert:\\
\begin{align*}
	p_\gamma &=	\mid   \overrightarrow{p}_\gamma\mid\\
	p_e &=	\mid   \overrightarrow{p}_e\mid\\
\end{align*}



\subsection{Für das masselose Photon gilt: }
\begin{align*}
	E_\gamma =\sqrt{m^2c^4+c^2p_\gamma^2}=cp_\gamma\quad \text{da m=0}
\end{align*}

\subsection{Für das ruhende Elektron gilt: }
\begin{align*}
	E_e =m_ec^2
\end{align*}
\subsection{Für das Photon nach dem Stoß  gilt: }
\begin{align*}
	E'_e =\sqrt{m_e^2c^4+c^2p_e^2}
\end{align*}
\subsection{So ergibt sich für die Energieerhaltung}
\begin{align}
	E_\gamma + E_e&= E'_\gamma + E'_e\nonumber\\
	cp_\gamma+m_ec^2&=cp'_\gamma+\sqrt{m_e^2c^4+c^2{p'_e}^2} \label{Energie}
\end{align}
Gesucht: Beziehung zwischen $p_\gamma, p'_\gamma$ und den Winkel $\theta$. Also muss $p'_e$ eliminiert werden.\\
\subsection{Für die Impulserhaltung gilt:}
Koordinatensystem geschickt gewählt:\\
\begin{itemize}
	\item Elektron im Ursprung
	\item Photon fliegt entlang der X-Achse direkt auf das Elektron
	
\end{itemize}
\begin{center}
	\begin{tikzpicture}[scale=1.5, >=latex]
		
		% Achsen
		\draw[->] (-0.2,0) -- (3.5,0) node[right] {$x$};
		\draw[->] (0,-2.2) -- (0,2.2) node[above] {$y$};
		
		% Ursprung = Elektron vor Stoß
		\filldraw[blue] (0,0) circle (0.04) node[below left] {Elektron (ruhend)};
		
		% Einfallender Photon
		\draw[->, thick, blue] (-2,0) -- (0,0);
		\node[blue] at (-2.2, 0.2) {Photon einfallend $\vec{p}_\gamma$};
		
		% Elektron nach Stoß
		\draw[->, thick, blue] (0,0) -- (2,-1);
		\node[blue] at (0.8,-0.6) {$\vec{p'}_e$};
		\node[blue] at (2.2,-1.1) {Elektron nach Stoß};
		
		% Photon nach Stoß
		\draw[->, thick, red] (0,0) -- (1.3,2) ;
		\node[red] at (0.6,1.3){$\vec{p'}_\gamma$};
		\node[red] at (1.5,2.1) {Photon nach Stoß};
		
		% Winkel Theta (zwischen x-Achse und gestreutem Photon)
		%\draw (1,0) arc (0:-26.565:1);
		%\node at (1.2,-0.3) {$\theta$};
		
		% Optional: Winkel des Elektrons (phi)
		\draw (0.7,0) arc (0:57.995:0.7);
		\node at (0.75,0.3) {$\theta$};
		\node[blue] at (-0.2,0.2) {$\vec{p}_e$};
	\end{tikzpicture}
\end{center}

\begin{align*}
	\overrightarrow{p_\gamma} +\overrightarrow{p_e}=&\overrightarrow{p'_\gamma}+\overrightarrow{p'_e}\\
\end{align*}
Daraus folgt dann:\\
\begin{align*}
	&\overrightarrow{p_\gamma}=\begin{pmatrix}p_\gamma\\0\\0 \end{pmatrix}\\
	&\overrightarrow{p_e}=\begin{pmatrix}0 \\0\\0 \end{pmatrix}\\
	&\overrightarrow{p'_\gamma}= \begin{pmatrix}p'_{\gamma x}  \\p'_{\gamma y}\\0 \end{pmatrix}\\
	&\overrightarrow{p'_e}=  \begin{pmatrix}p'_{ex} \\p'_{ey} \\0 \end{pmatrix}\\
\end{align*}
\begin{align*}
	\begin{pmatrix}p_\gamma\\0\\0 \end{pmatrix}+\begin{pmatrix}0 \\0\\0 \end{pmatrix}= \begin{pmatrix}p'_{\gamma x} \\p'_{\gamma y}\\0 \end{pmatrix}+  \begin{pmatrix}p'_{ex}\\p'_{ey} \\0 \end{pmatrix}\\
\end{align*}
Dies ergibt dann zwei Gleichungen:\\
\begin{align}
	p_\gamma=&p'_{\gamma x}+p'_{ex} \nonumber\\
	\Leftrightarrow \nonumber\\
	p'_{ex}&=p_\gamma-p'_{\gamma x} \label{pex}  \\
	0=&p'_{\gamma y}+p'_{e y} \nonumber\\
	\Leftrightarrow \nonumber\\
	p'_{ey}&=-p'_{\gamma y}\label{pe}
\end{align}



Ebenso gilt die Beziehung mit Pythagoras:\\
\begin{align}
	{p'}^2_e=	{p'}^2_{ex}+	{p'}^2_{ey}\label{p}
\end{align}
Jetzt (\ref{pex})  und (\ref{pe}) in (\ref{p}) einsetzen und dies ergibt dann die Gleichung:\\
\begin{align}
	{p'}^2&=(p_\gamma-p'_{\gamma x})^2+(-p'_{\gamma y})^2 \nonumber\\
	&=p_\gamma^2-2p_\gamma p'_{\gamma x}+p'^2_{\gamma x}+p'^2_{\gamma y}\label{ppp}
\end{align}

\begin{MathBox}[Komponenten der Impulserhaltung]
	Aus \(\vec p_\gamma=\vec p'_\gamma+\vec p'_e\) folgen komponentenweise 
	\(p_\gamma=p'_{\gamma x}+p'_{ex}\) und \(0=p'_{\gamma y}+p'_{ey}\).
	Mit Pythagoras erhält man \(p_e'^2=p_{ex}'^{\,2}+p_{ey}'^{\,2}\).
\end{MathBox}


Jetzt den Winkel $\theta$ mittels den $\cos(\theta)$ ausdrücken und einfügen. Es gilt ja:\\

\begin{center}
	\begin{tikzpicture}[scale=2, >=latex]
		
		% Achsen
		\draw[->] (-0.2,0) -- (3.5,0) node[right] {$x$};
		\draw[->] (0,-0.2) -- (0,2.2) node[above] {$y$};
		
		
		
		
		
		
		
		% Photon nach Stoß
		\draw[->, thick, red] (0,0) -- (1.3,2) ;
		\node[red] at (0.6,1.3){$\vec{p'}_\gamma$};
		\node[red] at (1.5,2.1) {Photon nach Stoß};
		
		% Winkel Theta (zwischen x-Achse und gestreutem Photon)
		%\draw (1,0) arc (0:-26.565:1);
		%\node at (1.2,-0.3) {$\theta$};
		
		% Optional: Winkel des Elektrons (phi)
		\draw (0.7,0) arc (0:57.995:0.7);
		\node at (0.75,0.3) {$\theta$};
		
	\end{tikzpicture}
\end{center}

\begin{align}
	p'_{\gamma x} = p'_\gamma \cos (\theta) \label{winkel}
\end{align}
\begin{NoteBox}[Projektion richtig anwenden]
	Nur die \emph{x}-Komponente ist \(p'_{\gamma x}=p'_\gamma\cos\theta\).
	Der Term \(p'^2_{\gamma x}+p'^2_{\gamma y}=p'^2_\gamma\) — nicht \(p'^2_{\gamma y}\) — 
	ist der Betragquadratsatz für den Photonenimpuls nach dem Stoß.
\end{NoteBox}


(\ref{winkel} ) in (\ref{ppp}) einsetzen und dies ergibt dann die Gleichung  (\ref{pp}) .\\
\begin{align}
	{p'}_e^2=p_\gamma^2-2p_\gamma p'_\gamma \cos (\theta)+p'^2_{\gamma y}\label{pp}
\end{align}
Hier nochmals die zwei Gleichungen (\ref{pp}) und (\ref{Energie}):\\
\begin{align*}
	{p'}_e^2&=p_\gamma^2-2p_\gamma p'_\gamma \cos (\theta)+p'^2_{\gamma y}\\
	cp_\gamma+m_ec^2&=cp'_\gamma+\sqrt{m_e^2c^4+c^2{p'_e}^2} 
\end{align*}
Die Gleichung \ref{pp} wird in \ref{Energie} eingesetzt.\\
\begin{align*}
	cp_\gamma+m_ec^2&=cp'_\gamma+\sqrt{m_e^2c^4+c^2(p_\gamma^2-2p_\gamma p'_\gamma \cos (\theta)+{p'}^2_{\gamma y})} \\
	cp_\gamma+m_ec^2-cp'_\gamma&=\sqrt{m^2_ec^4+c^2p^2_\gamma+c^2{p'}^2_{\gamma y}-2c^2p_\gamma p'_\gamma \cos(\theta)}\
\end{align*}
Gleichung quadrieren:
\begin{align*}
	(cp_\gamma+m_ec^2-cp'_\gamma)^2&=m^2_ec^4+c^2p^2_\gamma+c^2{p'}^2_{\gamma y}-2c^2p_\gamma p'_\gamma \cos(\theta)\\
\end{align*}
\begin{DidacticBox}[Quadratieren der Energiegleichung]
	Das Wurzelzeichen entkoppeln wir durch Quadrieren beider Seiten. 
	Wichtig: \emph{Ganze Gleichung} quadrieren und anschließend identische Terme 
	auf beiden Seiten streichen.
\end{DidacticBox}


Linke Seite ausmultiplizieren:
\begin{align*}
	c^2p^2_\gamma+m_e^2c^4+c^2{p'}^2_\gamma&+2m_ec^3p_\gamma-  2c^2p_\gamma{p'}_\gamma-2m_ec^3p'_\gamma=\\
	&m^2_ec^4+c^2p^2_\gamma+c^2{p'}^2_{\gamma y}-2c^2p_\gamma p'_\gamma \cos(\theta)\\
\end{align*}

Kürzen der gemeinsamen Termen:
\[
\begin{aligned}
	\cancel{\color{red}c^2 p_\gamma^2}
	+ \cancel{\color{blue}m_e^2 c^4}
	+ \cancel{\color{teal}c^2 {p'}_\gamma^2}	\;&
	+ \color{orange}2 m_e c^3 p_\gamma
	- \color{purple}2 c^2 p_\gamma p'_\gamma
	- \color{cyan}2 m_e c^3 p'_\gamma
	=\\[0.2em]
	\;&\cancel{\color{blue}m_e^2 c^4}
	+ \cancel{\color{red}c^2 p_\gamma^2}
	+ \cancel{\color{teal}c^2 {p'}_{\gamma y}^2}
	- \color{purple}2 c^2 p_\gamma p'_\gamma \cos\theta
\end{aligned}
\]



Nach dem Kürzen bleibt:

\[
2 m_e c^3 p_\gamma - 2 m_e c^3 p'_\gamma = -2 c^2 p_\gamma p'_\gamma \cos(\theta) + 2 c^2 p_\gamma p'_\gamma
\]

Faktorisiere jeweils \( 2 m_e c^3 \) und \( 2 c^2 p_\gamma p'_\gamma \):

\[
2 m_e c^3 (p_\gamma - p'_\gamma) = 2 c^2 p_\gamma p'_\gamma (1 - \cos\theta)
\]

Nun nach \( p_\gamma - p'_\gamma \) auflösen:

\[
p_\gamma - p'_\gamma = \frac{c^2}{m_e c^3} \cdot p_\gamma p'_\gamma (1 - \cos\theta)
= \frac{1}{m_e c} \cdot p_\gamma p'_\gamma (1 - \cos\theta)
\]






\subsection{Herleitung der Compton-Formel aus dem Impuls-Ausdruck}

Aus der Impuls-Gleichung nach dem Stoß gilt:
\[
p_\gamma - p'_\gamma = \frac{1}{m_e c} \cdot p_\gamma p'_\gamma (1 - \cos\theta)
\]

\subsection*{Zusammenhang zwischen Impuls und Wellenlänge}

Für Photonen (masselose Teilchen) gilt:
\[
E = h f = c p \quad \Rightarrow \quad p = \frac{h f}{c}
\]

Da \( c = \lambda f \), folgt:
\[
p = \frac{h}{\lambda}
\]



\subsection*{ Impulsdarstellung mit Wellenlänge}

Für Photonen setzen wir:
\[
p_\gamma = \frac{h}{\lambda}, \quad p'_\gamma = \frac{h}{\lambda'}
\]

Einsetzen in die Gleichung:
\[
p_\gamma - p'_\gamma = \frac{1}{m_e c} \cdot p_\gamma p'_\gamma (1 - \cos\theta)
\]
\[
\frac{h}{\lambda} - \frac{h}{\lambda'} = \frac{1}{m_e c} \cdot \frac{h}{\lambda} \cdot \frac{h}{\lambda'} (1 - \cos\theta)
\]

\subsection*{Gemeinsamer Nenner links}

\[
h \left( \frac{1}{\lambda} - \frac{1}{\lambda'} \right) = \frac{h^2}{m_e c \lambda \lambda'} (1 - \cos\theta)
\]

\subsection*{ Rechte Seite auf denselben Nenner bringen}

Multipliziere beide Seiten mit \( \lambda \lambda' \):

\[
h (\lambda' - \lambda) = \frac{h^2}{m_e c} (1 - \cos\theta)
\]

\subsubsection*{Durch \( h \) teilen}

\[
\lambda' - \lambda = \frac{h}{m_e c} (1 - \cos\theta)
\]

\subsection*{Endform: Die Compton-Formel}

\[
\boxed{
	\Delta \lambda = \lambda' - \lambda = \frac{h}{m_e c} (1 - \cos\theta)
}
\]


\begin{HistoryBox}[Comptons Experiment (1923)]
	A. H. Compton beobachtete bei Röntgenstrahlen eine winkelabhängige 
	Wellenlängenvergrößerung. Die Formel 
	\(\Delta\lambda=\tfrac{h}{m_ec}(1-\cos\theta)\) bestätigte den 
	Teilchenimpuls von Lichtquanten und war ein Meilenstein für die Quantentheorie.
\end{HistoryBox}


\begin{HypoBox}[Was wäre, wenn das Photon eine Masse hätte?]
	Eine (hypothetische) Ruhemasse des Photons würde \(E^2=c^2p^2+m_\gamma^2c^4\) implizieren 
	und die Compton-Verschiebung modifizieren. Präzisionsmessungen setzen extrem strenge 
	Grenzen auf \(m_\gamma\approx 0\).
\end{HypoBox}
