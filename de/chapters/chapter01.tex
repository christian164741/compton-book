\chapter{Einleitung}
\setcounter{section}{1}
\setcounter{subsection}{0}
\setcounter{subsubsection}{1}
\setcounter{secnumdepth}{3}
\setlength{\parindent}{0pt}

\section{Einleitung}


Als Arthur Compton im Jahr 1923 die Streuung von Röntgenstrahlung an Elektronen untersuchte, stieß er auf ein Resultat, das die klassische Wellentheorie des Lichts nicht erklären konnte. Die Wellenlänge der gestreuten Strahlung war größer als die der einfallenden – und die Differenz hing vom Streuwinkel ab. Dieses Experiment lieferte den ersten direkten Nachweis für den Impulsübertrag zwischen Photon und Elektron und bestätigte damit den Teilchencharakter des Lichts.

Der Compton-Effekt markiert einen Wendepunkt in der Physik: Er verbindet Energie- und Impulserhaltung auf quantenmechanischer Ebene und führt zur bekannten \textbf{Compton-Formel}, die den Wellenlängen-Shift präzise beschreibt. Die folgende Darstellung zeigt diese Herleitung Schritt für Schritt – historisch eingeordnet, mathematisch fundiert und physikalisch nachvollziehbar.

\vspace{1.2em}
\begin{quote}
	\small
	\textbf{Hinweis zur Methodik.}  
	Dieses Paper wurde in enger Zusammenarbeit zwischen Mensch und Künstlicher Intelligenz erstellt. 
	Die physikalische Argumentation und Struktur stammen vollständig vom Autor; die KI diente als Werkzeug zur Formulierung, Präzisierung und didaktischen Gestaltung. 
	Ziel ist es, zu zeigen, wie moderne KI-Unterstützung wissenschaftliche Arbeit nicht ersetzt, sondern bereichert – durch Klarheit, Struktur und Transparenz.
\end{quote}
\vspace{1.2em}