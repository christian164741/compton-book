\cleardoublepage

\renewcommand{\thesection}{\thechapter.\arabic{section}}
\renewcommand{\thechapter}{C}

\chapter{KI in der Wissenschaft – Werkzeug statt Wahrheit}
\phantomsection
\markboth{Anhang C}{Anhang C}
\label{anhangC:ki}

\subsection*{Motivation}
Dieses Buch entstand aus dem Wunsch, den Compton-Effekt und seine Rolle in der modernen Physik klar, fundiert und nachvollziehbar darzustellen. Dabei kam ein Werkzeug zum Einsatz, das zunehmend Teil wissenschaftlicher Arbeit wird: \textbf{künstliche Intelligenz}\index{Künstliche Intelligenz}, konkret das Sprachmodell \textbf{ChatGPT}\index{ChatGPT} von OpenAI.

Ziel dieses Anhangs ist es, offen zu legen, wie KI im Entstehungsprozess eingesetzt wurde – und warum sie \emph{unterstützen}, aber keine wissenschaftliche Autorenschaft ersetzen kann.

\subsection*{Was eine KI kann – und was nicht}
KI-gestützte Sprachmodelle können beim Schreiben und Strukturieren unterstützen. Sie helfen beim Formulieren, Glätten und Ordnen von Texten und können Denkanstöße für Gliederungen oder Erklärwege liefern.

Was KI nicht kann:
\begin{itemize}
	\item \textbf{physikalische Konzepte verstehen}, 
	\item \textbf{Herleitungen prüfen} oder bewerten,
	\item \textbf{Quellenkritik} ausüben,
	\item die \textbf{inhaltliche Verantwortung} übernehmen.
\end{itemize}

Damit gilt: KI ist nützlich – aber sie ist kein Ersatz für wissenschaftliches Denken.

\subsection*{Wie dieses Buch entstanden ist}
Die inhaltlichen Aussagen, Formeln, Herleitungen und Interpretationen stammen vollständig vom Autor. ChatGPT wurde eingesetzt für:
\begin{itemize}
	\item sprachliche Überarbeitung einzelner Passagen,
	\item Stilglättung und Textvarianten,
	\item Unterstützung bei Strukturierungsvorschlägen,
	\item Reflexion zur Verständlichkeit\index{Verständlichkeit}.
\end{itemize}

\textbf{Alle physikalischen Inhalte wurden selbst geprüft, überarbeitet oder verworfen.} KI hat keinen inhaltlichen Beitrag zur Argumentation oder Herleitung geleistet.

\subsection*{Ethische Fragen und Verantwortung}
Die Nutzung von KI wirft Fragen nach Autorschaft\index{Autorschaft}, Fehlern\index{Fehler} und Transparenz auf. Die Antwort folgt einem Grundprinzip wissenschaftlicher Arbeit: \textbf{Verantwortung}. Wer KI nutzt, bleibt verantwortlich für das Ergebnis – unabhängig davon, welche Formulierungen oder Vorschläge eine KI erzeugt hat.

KI wird damit zum Werkzeug, nicht zur Instanz wissenschaftlicher Wahrheit.

\subsection*{Empfehlungen für Forschung und Wissenschaft}
\begin{itemize}
	\item KI bewusst für Sprache und Struktur verwenden – nicht für Inhalte.
	\item Aussagen und Formeln immer selbst prüfen.
	\item Nutzung offen deklarieren, wenn sie relevant ist.
	\item KI nicht zur Täuschung\index{Täuschung}, sondern zur Verbesserung der Darstellung einsetzen.
\end{itemize}

\subsection*{Fazit}
Künstliche Intelligenz ist ein hilfreiches Werkzeug, wenn sie reflektiert und verantwortungsvoll eingesetzt wird. Die wissenschaftliche Arbeit selbst bleibt ein menschlicher Prozess: \textbf{kritisch, prüfend, methodisch}. Dieses Buch versteht sich als Beitrag zu einem aufgeklärten Umgang mit KI – nicht als Technikgläubigkeit, sondern als bewusste Nutzung moderner Werkzeuge.
